
Now, we would like to define the skyscraper sheaf on the \'{e}tale topology. Recall that in the case of a topological space $X$, a skyscraper sheaf $\mathcal{F}$ is such that for some $x\in X$, $\mathcal{F}_x=\Lambda$ for some abelian group $\Lambda$, while for all $y\neq x$, $\mathcal{F}_x=0$. This can be constructed as the sheafification of the presheaf
\[\mathcal{F}(U)=\left\{\begin{array}{rcl}\Lambda&\text{if}&x\in U,\\0&\text{if}&x\notin U;\end{array}\right.\]
which results in the direct sum of $\Lambda$ indexed on all the connected components of $U$.

\begin{definition}
%\cite[Section~I.6.Skyscraper~sheaves]{milne2013lectures}
Let $X$ be a variety over an algebraically closed field $k$, $x\in X$, and let $\Lambda$ be an abelian group. We define the \textbf{skyscraper sheaf} $\Lambda^x$ on $X_{\text{\'{e}t}}$ as follows: for any \'{e}tale map $\varphi:U\rightarrow X$, write
\[\Lambda^x(U)\coloneqq\bigoplus_{u\in\varphi^{-1}(x)}\Lambda.\]
\end{definition}

It can be checked that $\Lambda^x$ is indeed a sheaf on $X_{\text{\'{e}t}}$, and it is such that for $y\notin\varphi(U)$, $\Lambda^x(U)=0$.

\begin{remark}
Taking the stalk at $x$ is the left adjoint functor of taking the skyscraper sheaf of the abelian group. This means, there exists a natural bijection
\[\Hom(\mathcal{F},\Lambda^x)\simeq\Hom(\mathcal{F},\Lambda).\]
This tells us that to give a morphism of sheaves $\mathcal{F}\rightarrow\Lambda^x$, it is sufficient to give a morphism of abelian groups $\mathcal{F}\rightarrow\Lambda$.
\end{remark}

These sheaves are used very often in \'{e}tale cohomology. As said at the beginning of this chapter, \'{e}tale topology was firstly developed with the goal of doing cohomology of groups on varieties. That's why constructing sheaves out of abelian groups is so relevant.
